\documentclass[a4paper, 11pt]{article}
\usepackage{color}
\usepackage{fancyhdr}
\usepackage{float}
\usepackage{stfloats}
\usepackage{placeins}
\usepackage{tabularray}
\usepackage{xcolor,colortbl}
\usepackage[top=2.5cm, bottom=2cm, left = 2.5cm, right = 2.5cm]{geometry} 
\geometry{a4paper} 
\usepackage[utf8]{inputenc}
\usepackage{textcomp}
\usepackage{graphicx} 
\usepackage{amsmath,amssymb}  
\usepackage{bm}  
\usepackage[pdftex,bookmarks,colorlinks,breaklinks]{hyperref} 
\hypersetup{linkcolor=MSBlue,citecolor=black,filecolor=black,urlcolor=black} % black links, for printed output
\usepackage{memhfixc} 
\usepackage{pdfsync}  
\usepackage{xcolor}
\usepackage{titlesec}
\usepackage{tocloft}
\usepackage{rotating}

\definecolor{MSBlue}{RGB}{47, 84, 150}
\definecolor{MSGray}{RGB}{128, 128, 128}

\renewcommand{\cftsecfont}{\fontfamily{qag}\selectfont\bfseries} 
\renewcommand{\cftsecpagefont}{\fontfamily{qag}\selectfont\bfseries\color{MSBlue}} 
\renewcommand{\cfttoctitlefont}{\fontfamily{qag}\selectfont\LARGE\bfseries}               
\renewcommand{\familydefault}{phv}

\fancypagestyle{titlepage}{
  \fancyhf{}
  \rfoot{\fontfamily{qag}\fontsize{11pt}{0pt}\selectfont\color{MSGray} }
  \renewcommand{\headrulewidth}{0pt}
  \renewcommand\footrulewidth{0pt}
}



\pagestyle{fancy}
\renewcommand{\headrulewidth}{0pt}
\renewcommand{\footrulewidth}{0pt}
\setlength{\headheight}{15pt}
\rhead{\fontfamily{qag}\fontsize{10pt}{12pt}\selectfont\color{MSGray} \today}
\lhead[]{}
\fancyfoot[C]{\fontsize{10pt}{10pt}\selectfont\thepage} 



\titleformat{\section}
  {\fontfamily{qag}\selectfont\LARGE\bfseries\color{MSBlue}}
  {\thesection}{0.5em}{}
  
  
\titleformat{\subsection}
  {\fontfamily{qag}\selectfont\Large\mdseries\color{MSBlue}}
  {\thesubsection}{0.5em}{}

\titleformat{\subsubsection}
  {\fontfamily{qag}\selectfont\large\mdseries\color{MSBlue}}
  {\thesubsubsection}{0.5em}{}

\titlespacing\subsubsection{0pt}{12pt plus 4pt minus 2pt}{0pt plus 2pt minus 2pt}

\linespread{1.2} 

\begin{document}

\begin{titlepage}
  \thispagestyle{titlepage}

  \begin{center} 
    \end{center}


	\setlength{\parindent}{0pt}
	\vspace*{.15\textheight}
	\medbreak
	{\fontfamily{qag}\Huge\bfseries\color{MSBlue} Personal Development Report
    \par}
	\bigbreak
    \bigbreak
	{Michał Raczkowski\par}
    \smallbreak
    {\small AI-core-AI4-RB02 \par}
    \smallbreak
    {\small 4465024\par}
\end{titlepage}



\pagebreak


\tableofcontents




\pagebreak

\section{Introduction}
My name is Michael, a 21-year-old technology enthusiast with roots in Warsaw. I completed my secondary education with a VWO diploma in Milanówek, a place where the integration of technology in daily life sparked my curiosity and shaped my academic pursuits.
\medbreak
At the age of 14, my passion for technology manifested in a significant project — the creation of a robotic prosthetic limb for a teacher. This venture, which was brought to life by the time I was 16, involved the innovative use of Arduino, 3D printing, and open-source coding. It employed Electromyography (EMG) to interpret muscle movements, translating them into commands that enabled the prosthetic to move.
\medbreak

Following this success, I sought out internships that aligned with my interests, allowing me to delve deeper into 3D printing and data forensics. These experiences were invaluable, providing a practical foundation for my theoretical knowledge.
\medbreak

My academic journey continued at Fontys in Eindhoven, where I initially explored Mechatronics. However, my fascination with the intricate world of ICT and Software Engineering soon became apparent, prompting a shift in my focus. I find the complexities of ICT particularly compelling, as they offer endless possibilities for innovation and problem-solving.
\medbreak

Currently, in my quest for knowledge, I am keen to explore the convergence of coding with infrastructure. My objective is to expand my understanding of software systems and their underlying frameworks, while also enhancing my proficiency in hardware development.
\medbreak

In my personal time, I engage in designing PCBs and advancing my own tech-related projects. These activities not only fuel my creativity but also complement my academic and professional goals. Additionally, I have a keen interest in the evolution of gaming and enjoy exploring the historical context and development processes behind modern games. Culinary exploration is another pastime of mine, providing a delicious outlet for experimentation and discovery.

\section{Learning outcomes}

\subsection{Data Preparation \& Analysis}
You are able to aggregate and prepare given datasets as well as other (open) datasets and use them in data analysis and identify opportunities for predictive analytics.   
\medbreak
Aggregate means acquiring data from a variety of different sources and in different formats and putting it together into a meaningful larger total dataset. Prepare consists of cleaning the data according to theories of data quality, in such a way that the process of cleaning and preparing those data is repeatable, transparent to others, and the results are suitable for data analysis. Data analysis implies amongst others: descriptive analytics, statistical overviews, derived columns, trend analysis, etc. Opportunities for predictive analytics can be identified by finding correlations between features, principle component analysis, summarization, anomaly detection, etc. and include an impact forecast.  

\subsubsection{Rating - self assessment}
\textbf{Beginning} 
\subsubsection{Evidence}
Data according to sales were obtain from client, and were anonymization to protect comapny which provide data for persdonall challange. Sales data were obtain raw with thounsds of producths through last 4 years in monthly fasion. Data from all products were aggragated and sum of all sales in month. After proccesing it was exported to csv file. 
feature data about inflation and unemployment were obtain from national database of poland. Data was raw, and was provided yearly with vlaues per month, data was aggragated and put in to one file with sales data, also 



\subsection{Model Engineering}
You are able to use findings from data analysis to preprocess data, apply machine learning algorithms and evaluate the quality and 
\medbreak
Findings from data analysis implies that your choice of data sources and feature selection is based on opportunities for predictive analytics that you previously identified. Preprocess refers to applying systematic ways like feature selection, encoding, scaling, etc. of turning raw datasets into formats that are more suitable for model training. Apply consists of training of different types of models like classification, regression, etc., as well as tuning hyper-parameters. Evaluate means judging the results of machine learning with respect to recall, precision, accuracy, cross-validation, over/underfitted etc. A defined domain refers to the fact that your evaluation must address the problem and impact definition as given by the domain stakeholders, and evaluation metrics must be translated to be meaningful to them.

\subsubsection{Rating - self assessment}
\textbf{Beginning} 
\subsubsection{Evidence}
Various types of models were implemented (Decision tree regression, Random forest regression, Linear regression, Prophet) in personal project with purpose of predicting future sales. Performance of models were tested by trait-test split method with result of relatively satisfactory results. Models were also tested by cross-validation methods which are harder to apply when working with time series based data and models. Unformtmnatyl accuracy messured by corss validfation was not as accurat as assumptions, but it is coused becouse of nature of projrxt itslef and provided data. Forcating time-series based data is not easiest task, espacilay when our sales data lack of any seasonablity due to nature of product, bussines model of comapny, recent global events and economic crisys. Data for model was callected from client and from open sources, data was modified in way that in can be applied for model whihc is using client montlhy based data. Data was agregeted togeter into one data set to be esaly redble and accessible for model 



\subsection{Explainable AI}
You deliver AI projects that follow the three 'Explainable AI' principles of transparency, interpretability, and explainability. 
\medbreak
Transparency means that the process by which the used input data results in prediction models is reproducible, reliably described and its decisions are motivated. Interpretability addresses the possibility for humans to comprehend the project cohesion and results by making them comparable to the domain knowledge and baselines. Explainability refers to the application of tools and methods that turn black-box models into grey/white-box models by having the model draw out its decision making process and/or describe its feature importance. 

\subsubsection{Rating - self assessment}
\textbf{Undefined} 
\subsubsection{Evidence}

\subsection{Professional Standard}
You show that you conduct work in accordance with an industry supported methodological approach (AI Project Methodology) in terms of your project's goals, stakeholder involvement, applied research, decision making and reporting.
\medbreak
Goals refer to identified authentic immediate and long term issues that you work towards finding appropriate solutions for. Whilst defining your goals you explore the context and environment of your project and you make the necessary business, sustainable and ethical considerations. Stakeholder involvement implies that you involve relevant and competent partners in your project from beginning to the end. During your project you communicate constructively with all your stakeholders. Applied research implies that you effectively use research strategies and methods, like those in the DOT-framework, for your domain understanding and other research activities. Decision making means that you correctly identify the need for further iterations, using evaluation models like the TIC-tool and your stakeholder feedback. Reporting refers to well structured and well motivated, correct and relevant documents, using APA style referencing for used external sources, as well as using visualizations, concluding your project proportionally covering all four phases of the methodology.  

\subsubsection{Rating - self assessment}
\textbf{Orienting} 
\subsubsection{Evidence} 

\subsection{Personal Leadership}
You are aware of your strengths and pitfalls in ICT as well as your personal development. To nurture personal growth you are able to engage in actions that align with your core values, in a way that suits you.
\medbreak
Being aware of your strengths and pitfalls means that you are able to recognize (among other things through self-reflection and asking for feedback) what you are already good at and where growth is still possible.   Being able to engage in actions means that you take responsibility, growing towards a professional ICT practitioner, seeing and seizing opportunities in a structured, planned and efficient way. In a way that suits you means that in the activities you undertake you apply an approach that fits your style of acquiring knowledge and skills. 

\subsubsection{Rating - self assessment}
\textbf{Orienting} 
\subsubsection{Evidence}

\subsection{Internship Preparation}
You create chances to acquire and define an internship assignment based on a match between your ambitions, the school's requirements and the field of expertise related to your profile or specialisation

\subsubsection{Rating - self assessment}
\textbf{Beginning} 
\subsubsection{Evidence}



\section{Conclusion}


\section{Conclusion}


\end{document}