\documentclass[a4paper, 11pt]{article}
\usepackage{color}
\usepackage{fancyhdr}
\usepackage{float}
\usepackage{stfloats}
\usepackage{placeins}
\usepackage{tabularray}
\usepackage{xcolor,colortbl}
\usepackage[top=2.5cm, bottom=2cm, left = 2.5cm, right = 2.5cm]{geometry} 
\geometry{a4paper} 
\usepackage[utf8]{inputenc}
\usepackage{textcomp}
\usepackage{graphicx} 
\usepackage{amsmath,amssymb}  
\usepackage{bm}  
\usepackage[pdftex,bookmarks,colorlinks,breaklinks]{hyperref} 
\hypersetup{linkcolor=MSBlue,citecolor=black,filecolor=black,urlcolor=black} % black links, for printed output
\usepackage{memhfixc} 
\usepackage{pdfsync}  
\usepackage{xcolor}
\usepackage{titlesec}
\usepackage{tocloft}
\usepackage{rotating}

\definecolor{MSBlue}{RGB}{47, 84, 150}
\definecolor{MSGray}{RGB}{128, 128, 128}

\renewcommand{\cftsecfont}{\fontfamily{qag}\selectfont\bfseries} 
\renewcommand{\cftsecpagefont}{\fontfamily{qag}\selectfont\bfseries\color{MSBlue}} 
\renewcommand{\cfttoctitlefont}{\fontfamily{qag}\selectfont\LARGE\bfseries}               
\renewcommand{\familydefault}{phv}

\fancypagestyle{titlepage}{
  \fancyhf{}
  \rfoot{\fontfamily{qag}\fontsize{11pt}{0pt}\selectfont\color{MSGray} version 0.1v}
  \renewcommand{\headrulewidth}{0pt}
  \renewcommand\footrulewidth{0pt}
}



\pagestyle{fancy}
\renewcommand{\headrulewidth}{0pt}
\renewcommand{\footrulewidth}{0pt}
\setlength{\headheight}{15pt}
\rhead{\fontfamily{qag}\fontsize{10pt}{12pt}\selectfont\color{MSGray} \today}
\lhead[]{}
\fancyfoot[C]{\fontsize{10pt}{10pt}\selectfont\thepage} 



\titleformat{\section}
  {\fontfamily{qag}\selectfont\LARGE\bfseries\color{MSBlue}}
  {\thesection}{0.5em}{}
  
  
\titleformat{\subsection}
  {\fontfamily{qag}\selectfont\Large\mdseries\color{MSBlue}}
  {\thesubsection}{0.5em}{}

\titleformat{\subsubsection}
  {\fontfamily{qag}\selectfont\large\mdseries\color{MSBlue}}
  {\thesubsubsection}{0.5em}{}

\titlespacing\subsubsection{0pt}{12pt plus 4pt minus 2pt}{0pt plus 2pt minus 2pt}

\linespread{1.2} 

\begin{document}

\begin{titlepage}
  \thispagestyle{titlepage}
  \begin{center} 
    \end{center}


	\setlength{\parindent}{0pt}
	\vspace*{.15\textheight}
	\medbreak
	{\fontfamily{qag}\Huge\bfseries\color{MSBlue} Personal Development Report
    \par}
	\bigbreak
    \bigbreak
	{Michał Raczkowski\par}
    \smallbreak
    {\small AI-core-AI4-RB02 \par}
    \smallbreak
    {\small 4465024\par}
\end{titlepage}



\pagebreak


\tableofcontents

\vfill
\begin{table}[b]
  \centering
  \begin{tblr}{
    width = \linewidth,
    colspec = {Q[200]Q[133]Q[327]Q[248]},
    hlines,
    vlines,
  }
  \textbf{Version} & \textbf{Date} & \textbf{Author} & \textbf{Comment} \\
   0.1v                & 11.09.23             & M. Raczkowski   &
    \\

  \end{tblr}
\end{table}


\pagebreak

\section{Introduction}
My name is Michael, a 21-year-old technology enthusiast with roots in Warsaw. I completed my secondary education with a VWO diploma in Milanówek, a place where the integration of technology in daily life sparked my curiosity and shaped my academic pursuits.
\medbreak
At the age of 14, my passion for technology manifested in a significant project — the creation of a robotic prosthetic limb for a teacher. This venture, which was brought to life by the time I was 16, involved the innovative use of Arduino, 3D printing, and open-source coding. It employed Electromyography (EMG) to interpret muscle movements, translating them into commands that enabled the prosthetic to move.
\medbreak

Following this success, I sought out internships that aligned with my interests, allowing me to delve deeper into 3D printing and data forensics. These experiences were invaluable, providing a practical foundation for my theoretical knowledge.
\medbreak

My academic journey continued at Fontys in Eindhoven, where I initially explored Mechatronics. However, my fascination with the intricate world of ICT and Software Engineering soon became apparent, prompting a shift in my focus. I find the complexities of ICT particularly compelling, as they offer endless possibilities for innovation and problem-solving.
\medbreak

Currently, in my quest for knowledge, I am keen to explore the convergence of coding with infrastructure. My objective is to expand my understanding of software systems and their underlying frameworks, while also enhancing my proficiency in hardware development.
\medbreak

In my personal time, I engage in designing PCBs and advancing my own tech-related projects. These activities not only fuel my creativity but also complement my academic and professional goals. Additionally, I have a keen interest in the evolution of gaming and enjoy exploring the historical context and development processes behind modern games. Culinary exploration is another pastime of mine, providing a delicious outlet for experimentation and discovery.

\section{Feedback}

\subsection{Personal challenge}

\subsubsection{Feedback 5.10.23}
- If possible add new feature like temperature (or other e.g. Industrial sales)
- Choosing Model: start with SVM later experiment with Prophet or other efficient model for prediction (regression)
- If unable to find features just continue, because there are a lot of data from 01-2019 till 09-2023
- Because of large sample we can continue with less SVM with fewer features works worst but with large data we can achieve quite good result because its more efficient supervised learning
- On heat map features has good correlation
- On graph we also can see some correlation (e.g. sales and unemployment)
\smallbreak
 Snoeren,Jacco J.P.H.

\subsubsection{Feedback 31.10.23}
this appears not to be a full iteration: the proposal and domain understanding is missing, so it is impossible to understand the purpose and context of your challenge - e.g. a stakeholder is missing - at the end we can not see any accuracy score or confusion matrix - so the iteration seems to start half way and end half way - the AI project methodology is not followed
\smallbreak
~Welman, Nick N.P.M.

\subsection{Group Project}

\subsection{Exercises}

\subsubsection{3 Data Requirements}
Well done, this is already a very nice first version, with some good 'thinking outisde the box'! The major events are a nice creative idea, that could have a big impact, but think about how to transform that into data and how to link that to other data elements.

At this point, you might have more questions than answers, but you have create a nicely structured starting point. It helps you to identify assumptions, raise more in-depth questions and of course, to start listing and describing the data elements you need.

Looking at the headers, you have addressed most relevant aspects, although in next versions I would advise you to structure the data elements to a more condensed overview. Eventually, don't group the data elements by type, but by meaning/content. 
You did go into details already, that's really good. You might change/improve this when you progress in your project, but it is good to have thought about it already. it will help you to be more critical.
For the quality criteria, you tried to capture consistency and accuracy in a data requirement, but once you have started to explore data (sources) you might come up with an addition or change to this part. 

You have started to make an overview of data elements, and you look broader than just the traffic on the highway. That is good, but makes sure you can explain how it is connected to your target variable?). If you don't know if a requirement is valid or necesarry (yet), than add a comment or note and check or revise this in a later stage.
Everytime you explore or analyse the data, you might need to update the requirements as well.
For coming weeks: try to make the data elements more specific/detailed: think about how to to link the data elements (e.g. based on date/time, but maybe also other aspects).

In conclusion: this exercise was meant to help to to get started, and that is what you've done. of course, it is not a complete or perfect overview of data requirements yet; you have to keep working on that for the coming weeks, but as the start is sometimes the most difficult, you now have something to move forward from. Feel free to discuss further work or ideas with me!
\smallbreak
Schmitz, Niek N.T.A.

\subsubsection{1 Nearest Neighbors}
You didn't see correctly that the results are different because of the train/test split, although after the explanation I think you did. Do you know how to make them stable as well?

You performed the second exercise correctly. I see you had some issues with adding one feature, although this should not give an error message. We can discuss if you keep having the error.

You correctly conclude that changing n-neighbours gives different results and understand what's happening.
\smallbreak
Snoeren, Jacco J.P.H.

\section{Retrospect}
This couple of weeks were challenging, and my preformance wasn't the best as it could have been, but I still made progress, worked on my individual challenge, and worked with my group on our project, attends lectures and exercises to get most of it. I am aware I was behind schedule, but I am trying to catch up with everything, I lack right now.
\section{Conclusion}
My preformance have to improve but in my humble opinion I still made progress in KPI's I should have on the end of this semester 


\end{document}